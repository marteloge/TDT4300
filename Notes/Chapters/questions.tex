\chapter{Questions From the Lectures}

	\section{Forelesning 1}
		\begin{enumerate}
			\item Hva er datagruvedrift ("data mining")? Nevn eksempler på hva data mining er
			og hva det ikke er. 
			\item Hva er hovedhensiktene med data mining?
			\item Forklar hensiktene med klassifisering og klyngedannelse, samt forskjellene
			mellom disse. 
			\item Hva er datavarehus?
			\item Hvorfor er datavarehus nyttig? Nevn eksempler på datavarehusanvendelser.
			\item Hva er forskjellene mellom data mining og datavarehus?
			\item Forklar hvorfor vi trenger data cleaning og data integration..
		\end{enumerate}


	\section{Forelesning 2? Decision tree}

		Dr. Ola har en pasient som er veldig syk. Uten behandling vil han anslå at pasienten vil dø i løpet
		av tre måneder. Det eneste alternativet han har er en operasjon som er svært risikofull. Han vil klare
		å forlenge levetiden hans med ett år dersom operasjonen går bra. Det er imidlertid 30\% sjanse
		fir at han ikke vil overleve operasjonen. 

		\begin{enumerate}
			\item Tegn et decision tree for dette problemet. Vi sannsynlighetene for alle alternative
			utganger.
			\item Dr. Ola har underveis funnet ut at det finnes en mindre risikofull test så kan
			gjennomføres uten at pasienten nødvendigvis må dø. Denne testen vil da kunne anslå om pasienten
			vil overleve operasjonen eller kan dø uansett. DVS. Hvis testen er positiv øker sjansen for å 
			overleve operasjonen. Testen har følgende karakteristikker:
			 	\begin{enumerate}
			 		\item True-positiv (T) rate: sannsynligheten for testen vil være positiv dersom
			 		pasienten vil overleve operasjonen er 90\%.
			 		\item False-Positiv (F) rate: sannsynligheten for testen vil være positiv dersom pasienten
			 		ikke vil overleve operasjonen er 10\%. 
			 	\end{enumerate}
			Hva vil sannsynligheten fo rat pasienten vil overleve operasjonen dersom testen er positiv?
			\item Det viser seg likevel at testen vil kunne føre til fatale komplikasjoner. Dvs pasienten
			kan dø under testingen som for eksempel følge av allergiske reaksjoner, el.

			Tegn et nytt decision tree for å vise alternativene og konsekvensene for Dr. Olas problem.
		\end{enumerate}

	\section{Forelesning 3}

		\begin{enumerate}
			\item Forklar hvilke typer attributter det er
				\begin{enumerate}
					\item Hva er forskjellen på attributt og datatype
					\item Gi eksempler på kontinuerlige, discrete, og asymmetriske attributter.
				\end{enumerate}
			\item Hvorfor kan dimensjonalitet være et problem i data mining?
			\item Forklar hvordan dokumenter med teksinnhold kan gjøres om til numeriske data for 
			data mining.
			\item Hva menes med ordnede data? Gi eksempler.
			\item Hvorfor er datakvalitet viktig å håndtere? Gi eksempler på problemer som kan 
			oppstå 
				\begin{enumerate}
					\item Hva menes med outliners?
					\item Hvorfor kan outliners være et potensielt problem?
				\end{enumerate}
			\item Hvordan håndteres det dersom vi mangler noen verdier i  dataene våre?
			Forklar alternativene.
			\item Hvordan håndteres det dersom vi har duplikater i dataene våre? Forklar alternative løsninger.
			\item Forklar datapreprosseseringsteknikker som går an å bruke.
			\item Når brukes likhet og ulikhetsmål i data mining?
				\begin{enumerate}
					\item Hvilke mål har vi for disse?
					\item Gi eksempler.
				\end{enumerate}
		\end{enumerate}

	\section{Forelesning 4}
		
		\subsection*{Oppgave 1}

			Finn alle assosiasjonsregler i følgende database over studentenes fagvalg.
			Anta at minsup = 2 og minconf = 70\%.

			\begin{table}[H]
				\begin{tabular}{| l | l |}
					\hline
					{\bf SID} & {\bf Fag} \\ \hline
					1 & Matte, Fysikk, Kjemi \\ \hline
					2 & Fysikk, Kjemi, Norsk, Engelsk \\ \hline
					3 & Kjemi, Norsk \\ \hline
					4 & Matte, Fysikk, Norsk \\ \hline
					5 & Matte, Fysikk, Kjemi \\ \hline
				\end{tabular}
			\end{table}

		\subsection*{Oppgave 2}

			Hvordan ville du implementere kandidatementsettgenerering (what?) for:

			F2 = {{A, B}, {A,C} {B, C}, {B,D}} $\Rightarrow$ C3 = \{\{A, B, C\}\}

		\subsection*{Oppgave 3}

			\begin{enumerate}
				\item Lag et FPTree av følgende transaksjoner
				\item Gjør det samme, men nå uten sortering
			\end{enumerate}

			\begin{table}[H]
				\begin{tabular}{| l | l | }
					\hline
					1 & A, B, C, D, E \\ \hline
					2 & B, C, D, F \\ \hline
					3 & A, C, D, G \\ \hline
					4 & B, C, F \\ \hline
					5 & D, E, G, H, I \\ \hline
				\end{tabular}
			\end{table}

		\subsection*{Oppgave 4}
			Anta handlekurvdata nederst. Bruk apriori-algoritmen for å finne frekvente elementsett,
			git at minimum støtte er 22 \% (dvs. minimum support count er 2).

			\begin{table}[H]
				\begin{tabular}{| l | l | }
					\hline
						{\bf TID} & {\bf Kjøpte produkt} \\ \hline
						T100 & L1, L2, L5 \\ \hline
						T200 & L2, L4 \\ \hline
						T300 & L2, L3 \\ \hline
						T400 & L1, L2, L4 \\ \hline
						T500 & L1, L3 \\ \hline
						T600 & L2, L3 \\ \hline
						T700 & L1, L3 \\ \hline
						T800 & L1, L2, L3, L5 \\ \hline
						T900 & L1, L2, L3 \\ \hline
				\end{tabular}
			\end{table}

	\section{Forelesning 5}
		\begin{enumerate}
			\item Illustrer hvordan assosiasjonsregler (AR) ser ut.

			Hvilke anvendelsesområde brukes AR ?

			Hva er hovedhensiktene med AR?

			\item Hvorfor kan binærrepresentasjon av handlekurv være nyttig? Begrunn svaret.
			\item Forklar begrepene: elementsett (item set), støtteantall (support count),
			støtte (support), frekvent elementsett (frequent itemset), og konfidens (confidence).
			\item Forklar hvordan man formelt kan beregne og estimere støtte (s) og konfidens (c)
			dersom man har regelen X $\rightarrow$ Y.
			\item Forklar hva som menes med assosiasjonsregeloppdagelse og hvordan finner man den?

				\begin{enumerate}
					\item Hva er største problemet med "brute-force" metoden for
					assosiasjonsregeloppdagelse?
					\item Forklar et alternativ til Brute-Force.
				\end{enumerate}

			\item Forklar strategier for mer effektive regeloppdagelse.
				\begin{enumerate}
					\item Hva menes med pruning?
				\end{enumerate}
			\item Hva er Apriori-prinsippet? Hva er anti-monotonegenskap til support? 
			Hva er apriori-prinsippet godt for?
			\item Forklar hvordan apriori-algoritmen fungerer.
			\item Illustrer/Forklar hvordan man genererer kandidater for AR og pruning. 
		\end{enumerate}

	\section{Forelesning 6 - Classification part 1}

		\begin{enumerate}
			\item Hva er hovedhensiktene med klassifisering?
				\begin{enumerate}
					\item Gi eksempler for når klassifiseringen er nyttig (anvendelsesområde).
					\item Hvorfor trenger vi treningsdata til klassifiseringen?
				\end{enumerate}
			\item Hvilke klassifiseringsteknikker finnes det?
			\item Beslutningstre (Decision tree) er et eksempel på kassifiseringsmetode.
			Forklar hvordan denne metoden fungerer.
			\item Hvordan fungerer Hunt's algoritmen?
				\begin{enumerate}
					\item Forklar spesielt hvordan split utføres
					\item Vis hvordan man kan gjøre et multisplitt til binære splitt.
					\item Hvordan splitte kontinuerlige attributter?
				\end{enumerate}
			\item Forklar hvordan man velger den beste splittmetoden.
				\begin{enumerate}
					\item Forklar hvordan dette kan bli gjort vha GINI-metoden
					\item Forklar hvordan dette kan bli gjort vha Entropy-metoden
					\item Forklar hvordan dette kan bli gjort vha klassifiseringsfeil (classificatio error)
				\end{enumerate}
			\item Forklar når man stopper splittingen
			\item Hva menes med "underfitting" og "overfitting"?
				\begin{enumerate}
					\item Hva er årsaken til disse?
					\item Hvordan håndteres disse?
				\end{enumerate}
			\item Vis hvordan et nytt inslag kan få sin klasse
			\item Hvordan evalueres klassifiseringsmetoder?
		\end{enumerate}

	\section{Forelesning 7 - Classification part 2}

		\begin{enumerate}
			\item Hva menes med instansbasert klasifisering? Hvordan skiller dette seg fra 
			beslutningstre basert klassifisering?
			\item Hva er "Rote-learner"-klassifisering?
			\item Hva er "Nearest Neighbour"-klassifisering (kNN)?
				\begin{enumerate}
					\item Illustrer vha eksempel hvordan ideen bak dette
					\item Hvorfor trenger vi avstandsmål for å kunne bruke kNN?
					\item Hva er vektfaktor for kNN og når trenger vi å bruke det?
				\end{enumerate}
			\item Hva skjer i kNN når man bruker for liten k? For stor k?
			\item Man må ta hensyn til skala i forhold til data når man bruker avstandsmål i kNN.
			Hva betyr egentlig dette?
			\item Illustrer problemet med "Euclidiske"-avstanden
			\item Hva menes det med at kNN er en "Lazy learner"?
			\item Forklar prinsippet bak Bayes klassifikator (classifier)
			\item Forklar prinsippet bak SVM
		\end{enumerate}

	\section{Forelesning 8}

		\begin{enumerate}
			\item Hva er klynging? Hvordan er klynging forskjellig fra klassifisering? Nevn minst tre 
			eksemper for når klynging e rnyttig.
			\item Nevn tre representative algoritmer for klynging.
			\item Det er to hovedkategorier for klynging. Nevn disse og forklar hovedforskjellene. 
				\begin{itemize}
					\item Er det andre særpreg for klyngesett?
					\item Nevn fire klyngekarakteristikker? Forklar. 
				\end{itemize}
			\item Forklar hvordan K-means klyngingalgorimen fungerer.
				\begin{itemize}
					\item Hvorfor kan resultatklynger for et samme datasett bli forskjellige?
					\item Hvilke type avstandsmål kan brukes med k-means?
					\item Hva er kompleksiteten for k-means? Forklar 
				\end{itemize}
			\item Forklar hvordan klynging med k-means evalueres?
				\begin{itemize}
					\item Hva betyr det at klynging er optimal? sub optimal? 
				\end{itemize}
			\item Kan man bruke SSE til å vite verdien av k? 
			\item Har valg av initielle sentroider noe å si på kvaliteten på klynging? Begtunn svaret ditt. 
			\item Det at vi er nødt til å velge initielle sentroider gjør at vi ofte får dårlige resultat. 
			Hva kan man gjøre for å begrense problemene med dette?
			\item Hva menes med biscting k-means?
			\item Forklar prinsippet bak bisecting k-means. 
			\item Hva kan man gjøre med pre-postprosessering for k-means? Forklar. 
			\item Hvilke andre svakheter er det med k-means?
			\item Forklar hovedprinsippet bak hierarkisk klynging. Hva menes med at hierarkise klynging 
			kan visualiseres som dendrogram? 
			\item Forklar hvilke fordeler og ulemper er det med hierarkiskklynging
			\item Forklar to hovedtyper innenfor hierarkiskklynging
		\end{enumerate}

	\section{Forelesning 9}

		\begin{enumerate}
			\item Drøft kostnadene med hierarkisk klynging
			\item Forklar hvordan tekst kan klynges?
				\begin{enumerate}
					\item Kan k-means brukes? Kan HAC brukes?
					\item Hvilke avstansmål egne seg her?
					\item Hva enes med ICT? TF-ICF?
				\end{enumerate}
			\item Hvordan fungerer tetthestbaserte klynging?
				\begin{enumerate}
					\item Hvilke parametre må man beregne?
					\item Hva menes med kjernepunkt? Grensepunkt? Støypunkt? Bruk figur i forklaringen.
				\end{enumerate}
			\item Hva skjer når DBSCAN fungerer bra?
			\item Når fungerer DBSCAN ikke så bra?
			\item Hvordan påvirker valg av parametrene Eps og MinPts kvaliteten på DBSCAN
			klyngingen?
			\item Illustrer hvordan man velger passende verdier for Eps og MinPts
			\item Gitt DBSCAN-klynging, hvordan velges verdien av k?
			\item Hvorfan evalueres klynging?
				\begin{enumerate}
					\item Hva er utfordringen med klyngeevaluering?
					\item Hva er hensikten med klyngeevaluering?
				\end{enumerate}
			\item Hvilke aspekter må vi ta hensyn til i forhold til klyngevaliditet?
			\item Hvilke mål brukes til klyngevaliditet?
				\begin{enumerate}
					\item Hva er ekstern indeks?
					\item Hva er intern indeks?
					\item Hva er relativ indeks?
				\end{enumerate}
			\item Hva er silhuettoffisient?
			\item Forklar hvordan klyngevalidetet måle gjennom korrelasjon
		\end{enumerate}
